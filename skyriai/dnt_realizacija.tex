\section{Algoritmų realizacija}
\subsection{Dirbtinių neuroninių tinklų realizacija}
\label{sec:Dirbtinių neuroninių tinklų realizacija}

Norint geriau susipažinti su klasifikavimo klaidų kainų įvedimo metodais bei jų
savybėmis, prieš kuriant SM ir DNT kombinuojantį klasifikatorių, buvo nuspręsta atskirai
išsinagrinėti kainų įvedimo metodus į SM ir DNT klasifikatorius. Atlikus kainų
įvedimo į DNT metodų analizę (žr. ~\ref{sec:Perceptronas} skyrių) paaiškėjo,
kad geriausias duomenų subalansavimo metodas yra (\ref{eq:sigmoid}), o
praktikoje nusistovėjęs kainų įvedimo į DNT metodas yra (\ref{eq:perceptron output}),
kuris ir realiuotas šiame darbe.

...
