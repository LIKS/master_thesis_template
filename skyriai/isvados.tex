\vumifsectionnonum{Išvados}

Šiame darbe realizuota:
\begin{enumerate}
\item Sukurta bazinė C4.5 algoritmo realizacija ir keli jautrumo kainoms joje užtikrinimo metodai: gaubiamasis MetaCost algoritmas, pakeistosios klasių tikimybės, Laplace genėjimas. 
\end{enumerate}

...

Atlikus eksperimentus su sintetiniais ir realaus pasaulio duomenimis, gautos tokios išvados:
\begin{enumerate}
\item Parodyta, kad hibridinis kainoms jautrus klasifikatorius, paremtas \cite{Banerjee:1997:INN:267553.267554} hibridizacijos metodika ir jautrumo kainoms įvedimu į sprendimo medį bei daugiasluoksnį perceptroną atskirai, gali sumažinti inicializavusio sprendimo medžio klasifikavimo klaidų kainą su testiniais duomenimis. Taip pat parodyta, kad hibridas pasiekia geriausios kainos iteraciją greičiau nei analogiškos architektūros, tačiau atsitiktinių pradinių svorių daugiasluoksnis perceptronas.
\end{enumerate}

...
